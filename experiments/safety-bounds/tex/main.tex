\documentclass[12pt]{article}
\usepackage[margin=3cm]{geometry}
\usepackage[T1]{fontenc}
\usepackage[english]{babel}
\usepackage[utf8]{inputenc}
\usepackage{microtype}
\usepackage{mathtools}
\usepackage{amsthm}
\usepackage{xcolor}
\usepackage{nicefrac}

\theoremstyle{definition}
\newtheorem{definition}{Definition}
\newtheorem{observation}{Observation}
\newtheorem{simulation}{Simulation}

\usepackage{natbib}
\bibliographystyle{unsrtnat}


\usepackage[ruled,linesnumbered,vlined]{algorithm2e}
% \usepackage[ruled,linesnumbered,lined]{algorithm2e}
% bugfix (pkel reported in Oct 2022)
\renewcommand{\SetProgSty}[1]{\renewcommand{\ProgSty}[1]{\textnormal{\csname#1\endcsname{##1}}\unskip}}%
%
\SetKwProg{Fn}{function}{ is}{end}
\def\myargstyle{}
\SetProgSty{myargstyle}
\SetArgSty{myargstyle}
\SetKwComment{tcc}{// }{}
\def\mycommentstyle{\it\color{gray}}
\SetCommentSty{mycommentstyle}

\usepackage[hidelinks]{hyperref}

\begin{document}

\title{Safety bounds for Nakamoto Consensus}
\author{Patrik Keller}
\date{Draft, \today}
\maketitle

\section{Model}

We follow \citet{guo2022BitcoinLatency} who model the mining process as follows.
Blocks are numbered in the order they are mined.
A blockchain is a sequence $b_0, b_1, \dots, b_m$.
All chains start at genesis block $b_0 = 0$.
Proof-of-work establishes unique predecessors.
Each block also defines a blockchain, hence we use the terms interchangeably.
In the above example, block $b_m$ represents the complete chain $b_0, b_1, \dots, b_m$.
We define the height $h_b$ as the index of block $b$ in any blockchain containing~$b$.
%
The variable $t_b$ refers to the time when block $b$ was mined.
The mining delays $t_b - t_{b-1}$ are independent and exponentially distributed with rate~$\lambda$.
It follows that $t_{b_i} \leq t_{b_{i+1}}$ in the above example.

We divide the set of network participants into defenders and attackers.
Defenders mine defender blocks; attackers mine attacker blocks.
Each freshly mined block is an attacker block with probability $\alpha$ and a defender block otherwise.

We assume that defenders are honest.
They always mine on a chain of maximum height they are aware of.
When they mine a new block, they immediately share it with the other nodes.
We assume that the other defenders receive blocks within $\Delta$ time after they have been shared by a defender.

Attackers might mine on any chain.
They see all blocks immediately.
They can make visible attacker blocks to other participants at their willing with immediate effect.

\section{Version 0} \label{sec:v0}

Blocks which are mined within $\Delta$ after the last block are called \emph{tailgaters}.
For their \emph{rigged model}, \citet{guo2022BitcoinLatency} assume that defender tailgaters can be stolen by the attacker.
They convert all defender tailgaters to attacker blocks.
It is easy to show that the remaining defender blocks are all mined at a different height.
This property makes their remaining analysis tractable.

\begin{definition}[Tailgaters and Laggers]
  Block $b$ is called \emph{tailgater} if $t_b \leq t_{b-1} + \Delta$.
  Otherwise, if  $t_{b-1} + \Delta < t_b$, block $b$ is called a loner.
\end{definition}

\begin{definition}[Safety attack] \label{def:safety}
  At time $\tau$ the defenders learn about the target transaction~\texttt{tx}.
  At the same time, the attackers create a conflicting transaction~\texttt{tx'}.
  Any future blockchain will include \texttt{tx} or \texttt{tx'} but not both.

  Let $b$ be the first defender lagger mined at or after $\tau$.
  The defenders commit to one of the two transactions when their chain reaches height $h_b + k - 1$.
  The commitment may change when they learn a longer chain that includes the other transaction.

  The safety is violated if a defender commits to one of the transactions \texttt{tx} or \texttt{tx'} after another or the same defender committed to the other transaction.
\end{definition}

\begin{observation}[Stealing]
  Let $b$ and $b'$ be two defender blocks mined after $\tau$.
  Let $b'$ be a tailgater.
  It is possible that $b$ and $b'$ have the same height and one includes \texttt{tx}, the other \texttt{tx'}.
  We call this stealing because the attacker uses a defender block to extend the attacker chain.
\end{observation}

\begin{definition}[Geometric distribution]
  Success rate parameter $x$. Discrete support, number of trials $l$. Probability mass function $(1-x)^{l-1} \cdot x$.
\end{definition}

\begin{definition}[Modified geometric distribution] \label{def:modgeom}
  Honest mining rate parameter $p$, set success rate parameter $x = 1 - \nicefrac{1-p}p = \nicefrac{2p-1}p$.
\end{definition}

\begin{simulation}[Model \texttt{v0}] \label{sim:v0}
  Safety violations are feasible or infeasible depending on the realization of the mining process.
  Algorithm~\ref{alg:sim0} simulates one such realization under the assumption that the attacker can steal all defender tailgaters.
  It returns \texttt{Success} if a safety violation seems feasible and \texttt{Failure} otherwise.
  Counting the outcomes while repeatedly running the algorithm provides an increasingly accurate estimate for safety.
  We expect the results to match the bounds presented by \citet{guo2022BitcoinLatency}.
\end{simulation}

\begin{algorithm}
  \caption{Feasibility of safety violations; Version 0}
  \label{alg:sim0}

  \KwIn{$\tau, \alpha, \lambda, \Delta$; cutt-off $c$}
  \KwData{%
    Variable $t$ tracks the current time.
    Variable $a$ tracks the (maximal) height of the attacker chain.
    Variable $d$ tracks the height of the defender chain.
    Variable $s$ tracks the state of the target transaction.
  }

  $t \gets 0$; $a \gets 0$; $d \gets 0$; $s \gets \texttt{Pending}$\;
  \While{true}{
    \tcc{simulate proof-of-work}
    Sample $\delta$ from exponential distribution with rate $\lambda$\;
    $t \gets t + \delta$\;
    \tcc{handle mining event}
    \If{$ s = \texttt{Pending} $} {
      $a \gets \max(a, d)$\tcc*{attacker can adopt defender chain}
    }
    Sample $x$ from Bernoulli distribution with parameter $\alpha$\;
    \uIf(\tcc*[f]{attacker block or stolen tailgater}){$x$ \textbf{or} $\delta < \Delta$}{
      $a \gets a + 1$\;
    }
    \Else(\tcc*[f]{defender lagger}){
      $d \gets d + 1$\;
      \Switch(\tcc*[f]{include or commit target transaction}){$s$}{
        \uCase{\texttt{Pending} {\normalfont \textbf{with}} $t \geq \tau$}{ $s \gets \texttt{Included}(d)$ }
        \Case{$\texttt{Included}(h)$ {\normalfont \textbf{with}} $d \geq h + k - 1$}{ $s \gets \texttt{Committed}$ }
      }
    }
    \tcc{check termination}
    \uIf(\tcc*[f]{safety violation feasible}){$s = \texttt{Committed}$ \textbf{and} $a \geq d$}{
      \Return \texttt{Success}\;
    }
    \ElseIf{$s = \texttt{Committed}$ \textbf{and} $a < d - c$}{
      \tcc{
        Attacker has infinite time to catch up, but we cannot run these simulations forever.
        \citet{guo2022BitcoinLatency} argue that maximum reach of a random walk is geometrically distributed.
      }
      Sample maximum reach $x$ from modified geometric distribution (Def.\,\ref{def:modgeom}) with honest mining rate parameter $p = (1 - \alpha) e^{-\lambda\Delta}$\;
      \uIf(\tcc*[f]{safety violation feasible}){$x > d - a$}{
        \Return \texttt{Success}\;
      }
      \Else(\tcc*[f]{safety violation infeasible}){
        \Return \texttt{Failure}\;
      }
    }
  }
\end{algorithm}

\section{Version 1} \label{sec:v1}

Ling Ren mentioned a potential route for improving these bounds at AFT\,'22.
Assigning all defender tailgaters to the attacker is overly pessimistic.
It might be sufficient to drop them.
I will simulate such a scenario in this section.

\begin{observation}[Probabilities]\label{obs:ptick}
  Pre-rigging, a block is an attacker block with probability~$\alpha$ and a defender block with probability $\rho = 1 - \alpha$. A block is a lagger with probability $e^{-\lambda\Delta}$ and a tailgater otherwise.

  In the rigged model of \citet{guo2022BitcoinLatency}, defender tailgaters become attacker blocks.
  Post-rigging, a block is a defender block with probability $p = \rho e^{-\lambda\Delta}$ and an attacker block otherwise.

  We now drop defender tailgaters instead of assigning them to the attacker.
  The probability that a block is an attacker block is $\alpha$.
  The probability that a block is a defender block is $\rho e^{-\lambda\Delta}$.
  The probability that a block is dropped is $\rho (1 - e^{-\lambda\Delta})$.
  The probability that a non-dropped block is a defender block is
  \begin{align}
    p' := \frac{\rho e^{-\lambda\Delta}}{\alpha + \rho e^{-\lambda\Delta}}
  \end{align}
\end{observation}

\begin{simulation}[Model \texttt{v1}] \label{sim:v1}
  Algorithm~\ref{alg:sim1} closely resembles Algorithm~\ref{alg:sim0}.
  There are two differences. We ignore defender tailgaters instead of assigning them to the attacker and we change the parameter of the geometric distribution to reflect this change.
\end{simulation}

\begin{algorithm}
  \caption{Feasibility of safety violations; Version 1}
  \label{alg:sim1}

  \KwIn{$\tau, \alpha, \lambda, \Delta$; cutt-off $c$}
  \KwData{%
    Variable $t$ tracks the current time.
    Variable $a$ tracks the (maximal) height of the attacker chain.
    Variable $d$ tracks the height of the defender chain.
    Variable $s$ tracks the state of the target transaction.
  }

  $t \gets 0$; $a \gets 0$; $d \gets 0$; $s \gets \texttt{Pending}$\;
  \While{true}{
    \tcc{simulate proof-of-work}
    Sample $\delta$ from exponential distribution with rate $\lambda$\;
    $t \gets t + \delta$\;
    \tcc{handle mining event}
    \If{$ s = \texttt{Pending} $} {
      $a \gets \max(a, d)$\tcc*{attacker can adopt defender chain}
    }
    Sample $x$ from Bernoulli distribution with parameter $\alpha$\;
    \uIf(\tcc*[f]{attacker block}){$x$}{
      $a \gets a + 1$\;
    }
    \ElseIf(\tcc*[f]{defender lagger}){$\delta \geq \Delta$}{
      $d \gets d + 1$\;
      \Switch(\tcc*[f]{include or commit target transaction}){$s$}{
        \uCase{\texttt{Pending} {\normalfont \textbf{with}} $t \geq \tau$}{ $s \gets \texttt{Included}(d)$ }
        \Case{$\texttt{Included}(h)$ {\normalfont \textbf{with}} $d \geq h + k - 1$}{ $s \gets \texttt{Committed}$ }
      }
    }
    \tcc{check termination}
    \uIf(\tcc*[f]{safety violation feasible}){$s = \texttt{Committed}$ \textbf{and} $a \geq d$}{
      \Return \texttt{Success}\;
    }
    \ElseIf{$s = \texttt{Committed}$ \textbf{and} $a < d - c$}{
      \tcc{
        Attacker has infinite time to catch up, but we cannot run these simulations forever.
        \citet{guo2022BitcoinLatency} argue that maximum reach of a random walk is geometrically distributed.
      }
      Sample maximum reach $x$ from modified geometric distribution (Def.\,\ref{def:modgeom}) with honest mining rate parameter $p'$ from Obs.\,\ref{obs:ptick}\;
      \uIf(\tcc*[f]{safety violation feasible}){$x > d - a$}{
        \Return \texttt{Success}\;
      }
      \Else(\tcc*[f]{safety violation infeasible}){
        \Return \texttt{Failure}\;
      }
    }
  }
\end{algorithm}


\appendix

\section{Version 2} \label{sec:v2}

Simulate communication with event queue. Circumvent distinction of tailgaters and laggers.
Resembles the setting from Sect.~\ref{sec:v1} where all tailgaters are dropped.

\begin{definition} \label{def:ts}
  Let $D$ be the set of defenders,
  let $A$ be the set of attackers, and
  let $h_{n,t}$ be the height of the longest chain visible to $n \in D \cup A$ at time $t$.
  We define
  \begin{align}
    d_t & := \max_{n \in D} \, h_{n, t} ~,\\
    d'_t & := \min_{n \in D} \, h_{n, t} ~, \text{and}\\
    a_t & := \max_{n \in A} \, h_{n, t} ~.
  \end{align}
\end{definition}

\newcommand{\justbefore}[1]{{#1^{\circ}}}

\begin{definition}[Time step]
  Let $e_1, e_2, \dots$ enumerate in ascending order all the points in time when $d_\cdot, d'_\cdot$, or $a_\cdot$ change.
  Let $t = e_i$. We define $\justbefore{t}$ as a point in time satisfying $e_{i-1} < \justbefore{t} < t =  e_i$.
  Less formally, $\justbefore{t}$ happens just before $t$.
\end{definition}

\begin{observation}[Communication] \label{obs:communication}
  Defenders share all blocks, their communication is delayed by at most $\Delta$, hence for all $t$,
  \begin{align}
    d'_{t + \Delta} \geq d_t \,.
  \end{align}
\end{observation}

\begin{observation}[Defender mining] \label{obs:defendermining}
  Defenders always extend a chain of maximum height, hence for each defender block $b$,
  \begin{align}
    d_{t_b} \geq d'_\justbefore{t_b} + 1\,.
  \end{align}
\end{observation}

\begin{observation}[Attacker mining] \label{obs:attackermining}
  The attacker has to mine on an existing block, hence for each attacker block $b$,
  \begin{align}
    a_{t_b} \leq \max\left(d_\justbefore{t_b}, a_\justbefore{t_{b}}\right) + 1\,.
  \end{align}
\end{observation}

\begin{simulation}[Model \texttt{v2}] \label{sim:v2}
  Algorithm~\ref{alg:sim2} is an attempt to rephrase Algorithm~\ref{alg:sim1} with explicit communication.
  We do not categorize blocks in tailgaters and defenders.
\end{simulation}

\begin{algorithm}
  \caption{Feasibility of safety violations; Version 2}
  \label{alg:sim2}

  \small

  \KwIn{$\tau, \alpha, \lambda, \Delta$; cutt-off $c$}
  \KwData{%
    $Q$ is a time-ordered queue of future events.
    Variable $a$ tracks the (maximal) height of the attacker chain; $d$ and $d'$ track their time-indexed counterparts.
    Variable $s$ tracks the state of the target transaction.
  }

  $Q \gets \texttt{Empty}$; $a \gets 0$; $d \gets 0$; $d' \gets 0$; $s \gets \texttt{Pending}$\;
  Sample $\delta$ from exponential distribution with rate $\lambda$\;
  Insert \texttt{Proof-of-Work} into $Q$ at time $t + \delta$\;
  \While{true}{
    Pop $(t, e)$ from $Q$\tcc*{get current time and event from queue}
    \If{$ s = \texttt{Pending} $} {
      $a \gets \max(a, d)$\tcc*{attacker can adopt defender chain}
    }
    \Switch{e}{
      \uCase{\texttt{Proof-of-Work}}{
        Sample $x$ from Bernoulli distribution with parameter $\alpha$\;
        \uIf(\tcc*[f]{attacker block}){$x$}{
          $a \gets a + 1$\;
        }
        \Else(\tcc*[f]{defender block}){
          $d \gets \max(d, d' + 1)$\tcc*{extends shortest chain}
          \Switch{$s$}{
            \uCase{\texttt{Pending} {\normalfont \textbf{with}} $t \geq \tau$}{ $s \gets \texttt{Included}(d)$ }
            \Case{$\texttt{Included}(h)$ {\normalfont \textbf{with}} $d \geq h + k - 1$}{ $s \gets \texttt{Committed}$ }
          }
          Insert \texttt{Rx}($d$) into $Q$ at time $t + \Delta$\tcc*{delayed communication}
        }
        Sample $\delta$ from exponential distribution with rate $\lambda$\;
        Insert \texttt{Proof-of-Work} into $Q$ at time $t + \delta$\;
      }
      \Case{\texttt{Rx}($h$)}{
        $d' \gets \max(d', h)$\;
      }
    }
    \tcc{check termination}
    \uIf(\tcc*[f]{safety violation feasible}){$s = \texttt{Committed}$ \textbf{and} $a > d'$}{
      \Return \texttt{Success}\;
    }
    \ElseIf{$s = \texttt{Committed}$ \textbf{and} $a < d' - c$}{
      \tcc{
        Attacker has infinite time to catch up, but we cannot run these simulations forever.
        \citet{guo2022BitcoinLatency} argue that maximum reach of a random walk is geometrically distributed.
      }
      Sample maximum reach $x$ from modified geometric distribution (Def.\,\ref{def:modgeom}) with honest mining rate parameter $p'$ from Obs.\,\ref{obs:ptick}\;
      \uIf(\tcc*[f]{safety violation feasible}){$x > d' - a$}{
        \Return \texttt{Success}\;
      }
      \Else(\tcc*[f]{safety violation infeasible}){
        \Return \texttt{Failure}\;
      }
    }
  }
\end{algorithm}

\section{Version 3} \label{sec:v3}

Dropping all defender tailgaters is not realistic. Stealing is possible in some cases.

\begin{observation}[Feasibility of Stealing]
  Stealing is feasible if the defenders disagree about the state of the target transaction.
  There might be two blockchains of equal height, one including the target transaction, the other including a conflicting transaction.
  In these cases, dropping tailgaters gives unrealistic advantage to the defender.
\end{observation}

We introduce another time-indexed variable.
It tracks the feasibility of stealing.
We take inspiration from the safety analysis of parallel proof-of-work by \citet{keller2022ParallelProofofwork}.

\begin{definition} \label{def:ts2}
  We define $f_t = \texttt{False}$ if at time $t$ either all defenders prefer to include \texttt{tx} or all defenders prefer to include \texttt{tx'}. Otherwise, $f_t = \texttt{True}$.
\end{definition}

\begin{observation}
  While $f_t = \texttt{False}$, stealing is not possible.
\end{observation}

\begin{observation}\label{obs:sync}
  Let $b$ be a defender block.
  Assume $\tau < t_b$.
  Assume $a_\justbefore{t_b} \leq d_\justbefore{t_b} = d'_\justbefore{t_b}$.
  Assume $t_b + \Delta < t_{b+1}$.
  Then $f_{t_b + \Delta} = \texttt{False}$.

  \begin{proof}
    Block $b$ has height $d_{t_b} = d_\justbefore{t_b} + 1$.
    At time $t_b$, block $b$ is the only block of this height.
    This remains true until $t_b + \Delta$ because no other block is mined.
    At $t_b + \Delta$ all defenders know and prefer $b$.
  \end{proof}
\end{observation}

\begin{observation}
  Let $f_t = \texttt{False}$. All defenders prefer \texttt{tx} (resp. \texttt{tx'}).
  Then $f_t$ stays false until the attacker presents a chain of height $d'_t + 1$ or greater which includes \texttt{tx'} (resp. \texttt{tx}).
\end{observation}

\begin{simulation}[Model \texttt{v3}] \label{sim:v3}
  Algorithm~\ref{alg:sim3} closely resembles Algorithm~\ref{alg:sim2}.
  It adds tracking of $f_t$ and allows stealing while $f_t = \texttt{True}$.
\end{simulation}

\begin{algorithm}
  \caption{Feasibility of safety violations; Version 3}
  \label{alg:sim3}

  \small

  \KwIn{$\tau, \alpha, \lambda, \Delta$; cutt-off $c$}
  \KwData{%
    $Q$ is a time-ordered queue of future events.
    Variable $a$ tracks the (maximal) height of the attacker chain; $d$, $d'$, and $f$ track their time-indexed counterparts.
    Variable $s$ tracks the state of the target transaction.
  }

  $Q \gets \texttt{Empty}$; $a \gets 0$; $d \gets 0$; $d' \gets 0$; $f \gets \texttt{True}$, $s \gets \texttt{Pending}$\;
  Sample $\delta$ from exponential distribution with rate $\lambda$\;
  Insert \texttt{Proof-of-Work} into $Q$ at time $t + \delta$\;
  \While{true}{
    Pop $(t, e)$ from $Q$\tcc*{get current time and event from queue}
    \lIf(\tcc*[f]{attacker can adopt defender chain}){$ s = \texttt{Pending} $} {$a \gets \max(a, d)$}
    \lIf(\tcc*[f]{stealing becomes feasible}){$a > d'$} {$f \gets \texttt{True}$}
    \Switch{e}{
      \uCase{\texttt{Proof-of-Work}}{
        Sample $\delta$ from exponential distribution with rate $\lambda$\;
        Insert \texttt{Proof-of-Work} into $Q$ at time $t + \delta$\;
        Sample $x$ from Bernoulli distribution with parameter $\alpha$\;
        \uIf(\tcc*[f]{attacker block}){$x$}{
          $a \gets a + 1$\;
        }
        \Else(\tcc*[f]{defender block}){
          \If(\tcc*[f]{Observation~\ref{obs:sync}}){$t > \tau$ \textbf{and} $a \leq d = d'$ \textbf{and} $\delta > \Delta$}{
            $f \gets \texttt{False}$\;
          }
          $d \gets \max(d, d' + 1)$\tcc*{extends shortest chain}
          \lIf(\tcc*[f]{stealing}){$f$} {$a \gets \max(a, d' + 1)$}
          \Switch{$s$}{
            \uCase{\texttt{Pending} {\normalfont \textbf{with}} $t \geq \tau$}{ $s \gets \texttt{Included}(d)$ }
            \Case{$\texttt{Included}(h)$ {\normalfont \textbf{with}} $d \geq h + k - 1$}{ $s \gets \texttt{Committed}$ }
          }
          Insert \texttt{Rx}($d$) into $Q$ at time $t + \Delta$\tcc*{delayed communication}
        }
      }
      \Case{\texttt{Rx}($h$)}{
        $d' \gets \max(d', h)$\;
      }
    }
    \tcc{check termination}
    \uIf(\tcc*[f]{safety violation feasible}){$s = \texttt{Committed}$ \textbf{and} $a > d'$}{
      \Return \texttt{Success}\;
    }
    \ElseIf{$s = \texttt{Committed}$ \textbf{and} $a < d' - c$}{
      \tcc{
        Attacker has infinite time to catch up, but we cannot run these simulations forever.
        \citet{guo2022BitcoinLatency} argue that maximum reach of a random walk is geometrically distributed.
      }
      Sample maximum reach $x$ from modified geometric distribution (Def.\,\ref{def:modgeom}) with honest mining rate parameter $p'$ from Obs.\,\ref{obs:ptick}\;
      \uIf(\tcc*[f]{safety violation feasible}){$x > d' - a$}{
        \Return \texttt{Success}\;
      }
      \Else(\tcc*[f]{safety violation infeasible}){
        \Return \texttt{Failure}\;
      }
    }
  }

\end{algorithm}

\bibliography{zotero}
\end{document}
