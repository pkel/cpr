\bgroup
\newcommand{\height}{h}
\newcommand{\subheight}{h^*}

\newcommand{\subparents}{\operatorname{subparents}}

\section{George's Protocol}

George proposes to modify the $B_k$ protocol such that votes link back to the last seen (longest chain) vote.
In $B_k$, the votes link back to the most recent (longest chain) block.
George refers to his protocol as Tailstorm.
I have implemented George's protocol in the simulator and ran a few attacks.
I proceed with a description of my interpretation of Tailstorm.

Tailstorm is a family of protocols.
It is parametrized by $k \geq 2$.

\subsection{DAG validity}

Tailstorm data are called blocks.
The protocol defines a root block $g$.
All other blocks must inherit from $g$.

We associate each block with a height $\height$ and a sub height $\subheight$.
We set $\height(g) := 0$ and $\subheight(g) := 0$.
We use height and sub height to further restrict the validity of the Tailstorm DAG.

Blocks can be split in two categories: storm blocks and sub blocks.
Each sub blocks links to exactly one other block.
Sub blocks are not restricted by height or sub height.
If sub block $a$ was appended to parent block $b$, then we set $\height(a) := \height(b)$ and $\subheight(a) := \subheight(b) + 1$.

Each strong block links to exactly $k-1$ sub blocks.
All parent sub blocks must have the same height.
If strong block $a$ was appended to sub blocks with height $x$, then we set $\height(a) := x + 1$ and $\subheight(a) := 0$.

Each block, both storm and sub, has to fulfill a proof-of-work criterion.

\subsection{Protocol}

Nodes continuously grind on the proof-of-work criterion.

Nodes try to append blocks that maximize for height and sub height.
Let $b$ be the visible storm block of maximum height.
If there are $k-1$ sub blocks with height $\height(b)$, then a node tries to find a puzzle solution that allows him to append a strong block of height $\height(b) + 1$.
Otherwise, the node appends a sub block to the block with maximum height and sub height.
In case ambiguities, nodes prefer to append to the data that became visible first.

\subsection{Incentive Scheme}

We assign rewards to the individual nodes, depending on their contributions to the DAG.
The size and number of the rewards depends on the structure of the DAG.
The reward procedure is as follows.

Rewards are assigned backwards, storm block for storm block.
For each storm block $b$, calculate
\begin{align}
  r &= \frac1k \cdot \max_{a \in \parents(b)}\left(\subheight(a) + 1\right) \\
    &= c_1 + c_2 \cdot \max_{a \in \parents(b)}\subheight(a)
\end{align}
Then, for each $a$ in $\parents(b) \cup \{b\}$, reward $r$ to the node who appended $a$.

\subsection{Unauthorized Storm Blocks}

In the actual Tailstorm protocol draft, storm blocks do not have to fulfil a proof-of-work criterion.
According to my understanding, a storm block is just set of $k$ sub blocks or hashes thereof.
There is no possibility to grind on a storm block besides referencing other sub blocks.
If I follow correctly, the hash of a storm block $b$ is something like $\mathcal{H}(\operatorname{sort}(\parents(b))$.

Upcoming sub and storm blocks include such hash references.
Nodes have to de-reference the hashes to verify the validity of the reference.
I observe, that inverting $\mathcal{H}(\operatorname{sort}(\parents(\cdot))$ is hard if there are more sub blocks than necessary.
In order to avoid DoS-attacks, $\parents(b)$ has to be shared explicitly---at the latest with the next sub block that links to the strong block ($=\parents(b)$).

I argue that $\parents(b)$ can be merged into the next sub block.
I.\,e., make the strong block require a proof-of-work solution and count it as the first sub block of the next epoch.
This is what I have proposed above.

\egroup % protocol-local commands
